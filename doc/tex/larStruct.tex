\documentclass[a4paper,12pt]{article}
\usepackage[english]{babel}
\usepackage[latin1]{inputenc}
\setcounter{tocdepth}{2}
\begin{document}
\title{IN480 \hspace{0.5cm} Larstruct Module}
\author{D�sir�e Adiutori \and Alessia Giulia Cossu}
\maketitle
\tableofcontents
\newpage
\section*{Introduction}
\addcontentsline{toc}{section}{Introduction}
This module of LAR-CC library is about a hierarchical structures with LAR.
Hierarchical models of complex assemblies are generated by an aggregation of subassemblies, each one defined in a local coordinate system, and relocated by affine transformations of coordinates.

In this module there are:
\begin{itemize}
\item The \textbf{Affine transformations} are based on elementary matrices for affine transformation of vectors in any dimensional vector space, including translation, scaling and rotation;
\item The \textbf{Struct iterable class}, starting from an array representable geometric object, generates a new object representing the initial one in an alternative way, by means of specific fields attribution, e.g. body, box, etc..
\item The \textbf{structure to LAR conversion}
\item The \textbf{hierarchical complexes}
\end{itemize}

\newpage
\section{Implementation}
In questa sezione viene mostrata la traduzione, di alcune delle funzioni del modulo piu importanti, dal linguaggio python a julia.
\subsection{checkStruct}
\subsubsection{Conversion}
\textbf{Python}
\begin{verbatim}
def checkStruct(lst):
   obj = lst[0]
   if(isinstance(obj,tuple) or isinstance(obj,list)):
      dim = len(obj[0][0])
   elif isinstance(obj,Model): 
      dim = obj.n    
   elif isinstance(obj,Mat): 
      dim = obj.shape[0]-1    
   elif isinstance(obj,Struct): 
      dim = len(obj.box[0])    
   return dim
\end{verbatim}
\textbf{Julia}
\begin{verbatim}
function checkStruct(lst)
   obj = lst[1]
   if isa(obj,Matrix)
      dim=size(obj)[1]-1
   elseif(isa(obj,Tuple) || isa(obj,Array))
      dim=length(obj[1][1])
   elseif isa(obj,Struct)
      dim=length(obj.box[1])
   end
   return dim
end
\end{verbatim}
\subsubsection{Parallelization}
\begin{verbatim}
\end{verbatim}
\subsubsection{Unit-Test}
\begin{verbatim}
@testset "checkStruct" begin
  list=([[0.575,-0.175],[0.575,0.175],[0.925,-0.175],[0.925,0.175]],[[0,1,2,3]])
  @test checkStruct(list)==length(list[1][1][1])
  @test typeof(checkStruct(list))==Int
end
\end{verbatim}

\subsection{Traversal}
\subsubsection{Conversion}
\textbf{Python}
\begin{verbatim}
def traversal(CTM, stack, obj, scene=[]):
   for i in range(len(obj)):
      if isinstance(obj[i],Model): 
         scene += [larApply(CTM)(obj[i])]
      elif(isinstance(obj[i],tuple) or isinstance(obj[i],list)) and
         (len(obj[i]==2 or len(obj[i])==3):
         scene += [larApply(CTM)(obj[i])]
      elif isinstance(obj[i],Mat): 
         CTM = scipy.dot(CTM, obj[i])
      elif isinstance(obj[i],Struct):
         stack.append(CTM) 
         traversal(CTM, stack, obj[i], scene)
         CTM = stack.pop()
   return scene
\end{verbatim}
\textbf{Julia}
\begin{verbatim}
function traversal(CTM,stack,obj,scene=[])
   for i in range(1,len(obj))
      if isa(obj.body[i],Matrix)
         CTM=CTM*obj.body[i]
      elseif (isa(obj.body[i],Tuple) || isa(obj.body[i],Array)) && (length(obj.body[i])==2 || length(obj.body[i])==3)
         l=larApply(CTM)(obj.body[i])
         push!(scene,l)
      elseif isa(obj.body[i],Struct)
         push!(stack,CTM)	
         traversal(CTM,stack,obj.body[i],scene)
         CTM=pop!(stack)
      end
   end
   return scene
end
\end{verbatim}
\subsubsection{Parallelization}
\begin{verbatim}
\end{verbatim}
\subsubsection{Unit-Test}
\begin{verbatim}
@testset "traversal" begin
   square=([[0, 0], [0, 1], [1, 0], [1, 1]], [[0, 1, 2, 3]])
   @everywhere structure=Struct([square])
   @everywhere dim=checkStruct(structure.body)
   @test length(traversal(eye(dim+1),[],structure,[]))==length(structure.body)
   @test typeof(traversal(eye(dim+1),[],structure,[]))==Array{Any,1}
end
\end{verbatim}
\subsection{larApply}
\subsubsection{Conversion}
\textbf{Python}
\begin{verbatim}
def larApply(affineMatrix):
    def larApply0(model):
        if isinstance(model,Model):
            V = scipy.dot(array([v+[1.0] for v in model.verts]), affineMatrix.T).tolist()
            V = [v[:-1] for v in V]
            CV = copy.copy(model.cells)
            return Model((V,CV))
        elif isinstance(model,tuple) or isinstance(model,list):
            if len(model)==2: V,CV = model
            elif len(model)==3: V,CV,FV = model
            V = scipy.dot([list(v)+[1.0] for v in V], affineMatrix.T).tolist()
            if len(model)==2: return [v[:-1] for v in V],CV
            elif len(model)==3: return [v[:-1] for v in V],CV,FV
    return larApply0
\end{verbatim}
\textbf{Julia}
\begin{verbatim}
function larApply(affineMatrix)
   function larApply0(model)
      if length(model)==2
         V,CV=model
      elseif length(model)==3
         V,CV,FV = model
      end
      V1=Array{Float64}[]
      for (k,v) in enumerate(V)
         append!(v,[1.0])
         push!(V1,vec((v')*transpose(affineMatrix)))
         pop!(V[k])
         pop!(V1[k])
	   end
      if length(model)==2
         return V1,CV
      elseif length(model)==3
         return V1,CV,FV
      end
   end 
   return larApply0
end
\end{verbatim}
\subsubsection{Parallelization}
\begin{verbatim}
\end{verbatim}
\subsubsection{Unit-Test}
\begin{verbatim}
square=([[0, 0], [0, 1], [1, 0], [1, 1]], [[0, 1, 2, 3]])
cubes=([[0, 0, 0], [0, 0, 1], [0, 1, 0], [0, 1, 1], [1, 0, 0], [1, 0, 1], [1, 1, 0], [1, 1, 1]], [[0, 1, 2, 3, 4, 5, 6, 7]])

@testset "larApply Tests" begin
   @testset "2D" begin
      @testset "larApply Translation 2D" begin
      @test typeof(larApply(t(-0.5,-0.5))(square))==Tuple{Array{Array{Float64,N} where N,1},Array{Array{Int64,1},1}}
      @test larApply(t(-0.5,-0.5))(square)==([[-0.5, -0.5], [-0.5, 0.5], [0.5, -0.5], [0.5, 0.5]],[[0, 1, 2, 3]])
   end
      @testset "larApply Scaling 2D" begin
         @test typeof(larApply(s(-0.5,-0.5))(square))==Tuple{Array{Array{Float64,N} where N,1},Array{Array{Int64,1},1}}
         @test larApply(s(-0.5,-0.5))(square)==([[0.0,0.0],[0.0,-0.5],[-0.5,0.0],[-0.5,-0.5]],[[0,1,2,3]])
      end
      @testset "larApply Rotation 2D" begin
         @test typeof(larApply(r(0))(square))==Tuple{Array{Array{Float64,N} where N,1},Array{Array{Int64,1},1}}
         @test larApply(r(0))(square)==square
      end
   end

   @testset "3D" begin
      @testset "larApply Translation 3D" begin
         @test typeof(larApply(t(-0.5,-0.5,-0.5))(cubes))==Tuple{Array{Array{Float64,N} where N,1},Array{Array{Int64,1},1}}
         @test larApply(t(-0.5,-0.5,-0.5))(cubes)==([[-0.5, -0.5, -0.5], [-0.5, -0.5, 0.5], [-0.5, 0.5, -0.5], [-0.5, 0.5, 0.5], [0.5, -0.5, -0.5], [0.5, -0.5, 0.5], [0.5, 0.5, -0.5], [0.5, 0.5, 0.5]],[[0, 1, 2, 3, 4, 5, 6, 7]])
      end 
	
      @testset "larApply Scaling 3D" begin
         @test typeof(larApply(s(-0.5,-0.5,-0.5))(cubes))==Tuple{Array{Array{Float64,N} where N,1},Array{Array{Int64,1},1}}
         @test larApply(s(-0.5,-0.5,-0.5))(cubes)==([[0.0, 0.0, 0.0], [0.0, 0.0, -0.5], [0.0, -0.5, 0.0], [0.0, -0.5, -0.5], [-0.5, 0.0, 0.0], [-0.5, 0.0, -0.5], [-0.5, -0.5, 0.0], [-0.5, -0.5, -0.5]],[[0, 1, 2, 3, 4, 5, 6, 7]])
      end
	
      @testset "larApply Rotation 3D" begin
         @test typeof(larApply(r(pi,0,0))(cubes))==Tuple{Array{Array{Float64,N} where N,1},Array{Array{Int64,1},1}}
         @test larApply(r(pi,0,0))(cubes)[1]?[[0.0, 0.0, 0.0], [0.0, -1.22465e-16, -1.0], [0.0, -1.0, 1.22465e-16], [0.0, -1.0, -1.0], [1.0, 0.0, 0.0], [1.0, -1.22465e-16, -1.0], [1.0, -1.0, 1.22465e-16], [1.0, -1.0, -1.0]]
      end
   end
end
\end{verbatim}
\subsection{}
\subsubsection{Conversion}
\textbf{Python}
\begin{verbatim}

\end{verbatim}
\textbf{julia}
\begin{verbatim}

\end{verbatim}
\subsubsection{Parallelization}
\begin{verbatim}
\end{verbatim}
\subsubsection{Unit-Test}
\begin{verbatim}
\end{verbatim}

\section{Conclusion}


\end{document}
